% Created 2014-06-17 Tue 21:33
\documentclass[9pt,conference]{IEEEtran}
                     \linespread{0.7}
\usepackage[utf8]{inputenc}
\usepackage[T1]{fontenc}
\usepackage{fixltx2e}
\usepackage{graphicx}
\usepackage{longtable}
\usepackage{float}
\usepackage{wrapfig}
\usepackage{rotating}
\usepackage[normalem]{ulem}
\usepackage{amsmath}
\usepackage{textcomp}
\usepackage{marvosym}
\usepackage{wasysym}
\usepackage{amssymb}
\usepackage{hyperref}
\tolerance=1000
\author{lastland}
\date{\today}
\title{Notes}
\hypersetup{
  pdfkeywords={},
  pdfsubject={},
  pdfcreator={Emacs 24.3.1 (Org mode 8.2.5c)}}
\begin{document}

\section{exokernel}
\label{sec-1}
\subsection{problems of traditional OS abstraction}
\label{sec-1-1}
\begin{enumerate}
\item denies domain-specific optimizations.
\item discourages changes to implementations of existing abstractions.
\item restricts the flexibility.
\end{enumerate}
(general-purpose implementations of abstractions force applications that do not need a given feature to pay substantial overhead costs.)
\subsection{methods}
\label{sec-1-2}
secure binding, visible revocation, abort protocol

\emph{why not virtual machine?} severe performance penalty
\subsection{secure binding}
\label{sec-1-3}
decouples authorization from the actual use of a resource (seperate protection from management)
\begin{enumerate}
\item implementations:
\label{sec-1-3-1}
1, hardware mechanisims; 2, software caching; 3, downloading application code.
(\emph{benefits of downloading code}: 1, eliminate kernel crossings;
2, execution time of downloaded code can be readily bounded (no need to be scheduled))
\end{enumerate}
\section{flashvm}
\label{sec-2}
\subsection{problems with flash}
\label{sec-2-1}
\begin{itemize}
\item write amplification (rewrite multiple blocks). \emph{solution}: finer granularity write-back, stride prefetching, etc.
\item low reliability (a finite number of times to be written). \emph{solution}: page sampling, zero page sharing.
\item aging (fewer clean blocks). \emph{solution}: merged discard, dummy discard.
\end{itemize}

\subsection{performance}
\label{sec-2-2}
write locality because random reads are inexpensive;
more aggressive pre-cleaning; clustering;
much finer granularity page scanning;
prefetch a full set of valid pages;
stride prefetching for temporal locality.

\section{non-scalable lock}
\label{sec-3}
\subsection{model}
\label{sec-3-1}
\begin{itemize}
\item $a$: avg lock acq time on single core.
\item $a_k = (n - k) / a$: arrival rate.
\item $s$: time in serial section.
\item $c$: time for home dir to respond to a cache line req.
\item $s_k = 1 / (s + ck / 2)$: service rate.
\end{itemize}
If a large number of waiters, hard to go back.
$s_k$ rapidly decays as $k$ grows, for short serial section. (ans to why so rapidly)
\section{learning from mistakes}
\label{sec-4}
\subsection{bug pattern}
\label{sec-4-1}
1, Dead lock. 2, Atomicity violation. 3, Order violation. 4, Others.
\subsection{fixing strategies}
\label{sec-4-2}
\begin{itemize}
\item non-deadlock: condition check(while-flag; consistency check); code switch; design change; lock (add/change; adjust cric sec regions); others.
\item deadlock: give up resource; split resource; change acq order.
\end{itemize}
\section{a file is not a file}
\label{sec-5}
\subsection{findings}
\label{sec-5-1}
\begin{itemize}
\item \textbf{a file is not a file}: file -> small FS containing subfiles.
\item \textbf{sequential access is not sequantial}: \emph{pure} is rare (substantial according to 4.2.2). more 95\% sequential.
\item \textbf{auxiliary files dominate}: helper files
\item \textbf{writes are often forced}: most explicitly synced (some frequently).
\item \textbf{renaming is popular}: atomic operations are common, generally \emph{rename}.
\item \textbf{multiple threads perform I/O}: virtually all from a number of threads (to hide long-latency operations from interactive users).
\item \textbf{frameworks influence I/O}: the behavior of the framework, not just the application, determines I/O patterns.
\item wide variety of file types, mostly multimedia files.
\item apps tend to open many very small files, while most of the bytes accessed are in large files.
\item preallocation is used rarely, or in useless way\ldots{}
\end{itemize}

\section{lfs}
\label{sec-6}
\subsection{structures}
\label{sec-6-1}
\begin{enumerate}
\item inode map:
\label{sec-6-1-1}
Current location of each inode.
\uline{Blocks are written to log; addresses of blocks in checkpoint region.}
Almost always cached in main memory.
\item segment usage table:
\label{sec-6-1-2}
1, the number of live bytes in the seg.
2, most recent modified time of any block in the seg.
Used by cleaner.
\uline{Blocks are written to log, addresses of blocks in checkpoint region.}
\item checkpoints:
\label{sec-6-1-3}
Special fixed position on disk.
Addresses of all the blocks in inode map, seg usage table, current time, pointer to last seg written.
Two checkpoint regions, operations alternate between them.
Time: perriodically, when FS unmounted, system shut down.
\item directory operation log:
\label{sec-6-1-4}
Operation code, location of dir entry (inum and pos within dir), contents (name and inum), new ref count.
In log, before corresponding dir block or inode.
\end{enumerate}

\section{device driver}
\label{sec-7}
\subsection{redundacy}
\label{sec-7-1}
many opportunities. \emph{methods}: 1. procedural abstractions; 2. better multiple chipset support; 3. table driven programming.
% Emacs 24.3.1 (Org mode 8.2.5c)
\end{document}